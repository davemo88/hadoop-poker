%%%%%%%%%%%%%%%%%%%%%%%%%%%%%%%%%%%%%%%%%
% Journal Article
% LaTeX Template
% Version 1.3 (9/9/13)
%
% This template has been downloaded from:
% http://www.LaTeXTemplates.com
%
% Original author:
% Frits Wenneker (http://www.howtotex.com)
%
% License:
% CC BY-NC-SA 3.0 (http://creativecommons.org/licenses/by-nc-sa/3.0/)
%
%%%%%%%%%%%%%%%%%%%%%%%%%%%%%%%%%%%%%%%%%

%----------------------------------------------------------------------------------------
%	PACKAGES AND OTHER DOCUMENT CONFIGURATIONS
%----------------------------------------------------------------------------------------

\documentclass[twoside]{article}

\usepackage{lipsum} % Package to generate dummy text throughout this template

\usepackage[sc]{mathpazo} % Use the Palatino font
\usepackage[T1]{fontenc} % Use 8-bit encoding that has 256 glyphs
\linespread{1.05} % Line spacing - Palatino needs more space between lines
\usepackage{microtype} % Slightly tweak font spacing for aesthetics
\usepackage{graphicx}
\graphicspath{ {images/} }
\graphicspath{ {figures/} }

\usepackage[hmarginratio=1:1,top=32mm,columnsep=20pt]{geometry} % Document margins
\usepackage{multicol} % Used for the two-column layout of the document
\usepackage[hang, small,labelfont=bf,up,textfont=it,up]{caption} % Custom captions under/above floats in tables or figures
\usepackage{booktabs} % Horizontal rules in tables
\usepackage{float} % Required for tables and figures in the multi-column environment - they need to be placed in specific locations with the [H] (e.g. \begin{table}[H])
\usepackage{hyperref} % For hyperlinks in the PDF

\usepackage{lettrine} % The lettrine is the first enlarged letter at the beginning of the text
\usepackage{paralist} % Used for the compactitem environment which makes bullet points with less space between them

\usepackage{abstract} % Allows abstract customization
\renewcommand{\abstractnamefont}{\normalfont\bfseries} % Set the "Abstract" text to bold
\renewcommand{\abstracttextfont}{\normalfont\small\itshape} % Set the abstract itself to small italic text

\usepackage{titlesec} % Allows customization of titles
\renewcommand\thesection{\Roman{section}} % Roman numerals for the sections
\renewcommand\thesubsection{\Roman{subsection}} % Roman numerals for subsections
\titleformat{\section}[block]{\large\scshape\centering}{\thesection.}{1em}{} % Change the look of the section titles
\titleformat{\subsection}[block]{\large}{\thesubsection.}{1em}{} % Change the look of the section titles

\usepackage{fancyhdr} % Headers and footers
\pagestyle{fancy} % All pages have headers and footers
\fancyhead{} % Blank out the default header
\fancyfoot{} % Blank out the default footer
%%AH\fancyhead[C]{Running title $\bullet$ November 2012 $\bullet$ Vol. XXI, No. 1} % Custom header text
%%\fancyhead[C]{November 18th, 2014}
\fancyfoot[RO,LE]{\thepage} % Custom footer text

%----------------------------------------------------------------------------------------
%	TITLE SECTION
%----------------------------------------------------------------------------------------

\title{\vspace{-15mm}\fontsize{24pt}{10pt}\selectfont\textbf{Hadoop Poker: Machine Learning in Texas Hold'em}} % Article title

\author{
\large
\textsc{Adrienne Humblet, David Kasofsky}\\[2mm] 
\normalsize New York University \\ 
\date{}
\vspace{-5mm}
}


%----------------------------------------------------------------------------------------

\begin{document}

\maketitle{} % Insert title

\thispagestyle{fancy} % All pages have headers and footers

%----------------------------------------------------------------------------------------
%	ABSTRACT
%----------------------------------------------------------------------------------------

\begin{abstract}

\noindent {
No Limit Hold 'Em (NLHE) is a fun recreational game for many around the world and a TV sensation. This project develops a tool for analyzing the simplest decisions in NLHE, namely when the player when has opportunity to make the first move in a hand. We use hand histories, i.e. the logs of game actions in poker, to train SVMs to classify poker situations. This tool helps players answer the question: “Should I play this hand?”
} 

\end{abstract}

%----------------------------------------------------------------------------------------
%	ARTICLE CONTENTS
%----------------------------------------------------------------------------------------

\begin{multicols}{2} % Two-column layout throughout the main article text

\section{Introduction}
An online poker site, e.g. PokerStars.com REF, will easily play billions of hands per year REF and a professional player may play hundreds of thousands if not millions of hands a year REF. Each of these hands is logged as part of what is called a hand history.

Hand histories are useful for both players and poker game organizers. Players may wish to improve their play whereas organizers, e.g. PokerStars, may wish to observe how players act in order to offer the most attractive games or promotions. A single professional player provided ten months of hand histories to us and it was about a gigabyte of text. Thus a single poker site could easily generate many terabytes of hand histories each year and so Hadoop is an appropriate platform for hand history analysis. Furthermore, hand history analysis is essentially log file analysis, a common use case for technology like Hadoop REF.

We focus on one of the most popular forms of poker: No Limit Texas Hold 'Em (NHLE) tournaments REF. One can find a detailed description at REF. In particular we concern ourselves with the simplest situations in the game. Poker hands vary significantly in their complexity and so we limit ourselves to the cases in which a player has the opportunity to make the first bet. We use support vector machines (SVMs) REF to predict when a player should enter the hand and use a professional player's hand history as the bar for good performance.

%The goal of this analytic is to extract and present information from online poker hand logs. Our analysis would be restricted to No Limit Texas Hold'em Tournaments, an extremely popular form of poker. While much research has been devoted to post-flop strategy we chose to focus on pre-flop strategy. Specifially, we focused on what a player's action should be when they are in a position to be first to act. In other words, if all previous players have either folded or checked, what move should a player perform based on his or her cards, bankroll, and the bankrolls of the other players? The position of first to act is an especially interesting one because the player has more leverage for bluffing or setting the tone of a game. 

% Dummy text

%------------------------------------------------

\section{Motivation}

Poker is an interesting game to study for several reasons. First, it is a multi-billion dollar industry, e.g. PokerStars.com was purchased for almost \$5 billion in 2014 REF and is only one of hundreds of online poker sites REF. Second, it is a classical game of logic, deception and mathematics, requiring a basic knowledge of probability and game theory. Third, many consider poker a model game for the development of machine learning, such as the University of Alberta Computer Poker Research Group (CPRG) REF. There is much analysis surrounding poker academically, e.g. CPRG, and commercially, e.g. analytical poker software like REF REF and REF.

Our focus, NLHE tournament hand histories, is warranted for at least the following reasons:

\begin{compactitem}
	\item Characterizing individual players. If one gathers sufficient data about a specific player, patterns in that player's play may become identifiable and thus exploitable when playing against that player. A player could also examine themselves to find where they are losing the most value.
	\item Identifying profitable strategies. By examining many hands, one can identify which strategies tend to be profitable. Here it is critical to have a notion of the similarity of poker hands since identical situations will virtually never arise.
\end{compactitem}

We focus on the second reason by analyzing the simplest poker situations, as we mentioned above. Limiting the 

Imperfect information.
Leveraging the opportunity to be first to act. 

%------------------------------------------------

\section{Related Work}

The CPRG has published many papers related to our topic.
One such paper \cite{SVMPoker} describes using a support vector machine to train a poker bot on post-flop strategy. Since this yielded effective results, our project aims to extend a similar SVM mechanism for pre-flop strategy.

Others, such as \cite{holdemml}, have used machine learning to build poker bots. SVMs and other classification algorithms are used to select an action to take based on a given situation.

None of the research we encountered focused on pre-flop strategy. 

%------------------------------------------------

\section{Design}


\subsection{Sources}
We used two sources of hand histories.
\begin{compactitem}
\item{Poker Hand History} - Our first source is the private hand history of professional poker player Jaime Staples who kindly agreed to donate his data to us. Jaime Staples is a 23 year old Canadian with six figures of NLHE tournament cashes.
\item{IRC Poker Database} - Seven years of poker data scraped from a dozen IRC poker channels. It was compiled and made available by the CPRG.
\end{compactitem}

\subsection{Feature Vectors}
Our feature vectors consists of the action taken (fold or call/raise), number of players, player's position, all players' bankrolls, and the strength of the players cards. One of our biggest challenges was finding a feature vector that was possible based on both of our datasets. For example, the Jamie Staples data showed the players hands at all times whereas the IRC data did not. Similarly, the Jamie Staples data showed the potsize at the flop whereas the IRC data did not.  

We normalized all the bankrolls according to the largest bankroll at the table. 

\subsection{Design}
Using map reduce, we standardized both of our data sets to a common format and merged them. Next, we ran the joint dataset into another map reduce that translated the player's cards, e.g. AcKs, into a hand strength value based on a table of poker hand probabilities. We used the resulting data set to train an SVM using Spark and to analyze the data using Hive. 

\includegraphics[width=1\columnwidth]{Flowchart.png}
Figure1.png


%------------------------------------------------

\section{Results}

%--------------------------------------------

\section{Future Work}

%------------------------------------------------

\section{Conclusion}

%------------------------------------------------

\section{Acknowledgements}

Many thanks to Jaime Staples for sending us his hand histories free of charge. 


%----------------------------------------------------------------------------------------
%	REFERENCE LIST
%----------------------------------------------------------------------------------------

\begin{thebibliography}{99} % Bibliography - this is intentionally simple in this template
\bibitem{SVMPoker} J Pfund. Support Vector Machines in the Machine Learning: Classifier for a Texas Hold'em Poker Bot.  University of Pennsylvania, 2007.

\bibitem{clustering} N. Bard, D. Nicholas, C. Szepesvári, and M. Bowling. Decision-theoretic Clustering of Strategies. University of Alberta. 2015

\bibitem{holdemml} L Teofilo, L Reis. HoldemML: A Framework to generate No Limit Hold’em Poker Agents from Human Player Strategies. Conferencia Iberica de Sistemas Tecnologia de Informacao. 2011.

\bibitem{aymmsetric abstractions} N Bard, M Johanson, M Bowling. Asymmetric Abstractions for Adversarial Settings. Proceedings of the Thirteenth International Conference on Autonomous Agents and Multi-Agent Systems (AAMAS). May 2014.
\end{thebibliography}

%----------------------------------------------------------------------------------------

\end{multicols}

\end{document}
